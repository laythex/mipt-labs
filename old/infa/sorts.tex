\documentclass[14pt, a4paper]{article}
\usepackage[utf8]{inputenc}
\usepackage[russian]{babel}
\usepackage{graphicx}
\usepackage{listings}

\title{\textbf{Отчет о выполнении лабы по сортам}}
\author{Калашников Михаил, Б03-205}
\date{}

\begin{document}

\maketitle

\begin{enumerate}
\setcounter{enumi}{-1}

\item Пузырек и его товарищи

Мною были реализованы сортировки пузырьком, вставкой и перемешиванием. Построив график зависимости времени сортировки от размера исходного массива в двойном логарифмическом масштабе, получим прямую. Нетрудно заметить, что $\frac{log(t)}{log(N)}=2$. Из этого следует, что изначальная зависимость была квадратичной.

\begin{figure}[!h]
\centering
\includegraphics[scale=0.7]{0.png}
\label{image0}
\end{figure}

\clearpage
\item Пузырек, но быстрее

Проведем указанные манипуляции с компилятором и построим зависимости.

\begin{figure}[!h]
\centering
\includegraphics[scale=0.7]{1.png}
\label{image1}
\end{figure}

\clearpage
\item А теперь настоящие быстрые сортировки

Я выбрал сортировку кучей, расческой и слиянием. Построив график в осях $\frac{t}{NlogN}$ и $N$, получается значение, близкое к константе, начиная с некоторого размера массива.

\begin{figure}[!h]
\centering
\includegraphics[scale=0.7]{2.png}
\label{image2}
\end{figure}

\clearpage
\item $O(N^2)$ vs $O(NlogN)$

Поместим зависимости из пунктов 0 и 2 на один график.

\begin{figure}[!h]
\centering
\includegraphics[scale=0.7]{3.png}
\label{image3}
\end{figure}

\clearpage
\item Зависимость от начальных данных

Будем скармливать каждому из шести алгоритмов различные массивы: отсортированные, неотсортированные и отсортированные в обратном порядке. Полученные 18 засимостей сгруппируем и поместим на шесть графиков.

\begin{figure}[!h]
\centering
\includegraphics[scale=0.7]{4.png}
\label{image4}
\end{figure}


\end{enumerate}

\end{document}