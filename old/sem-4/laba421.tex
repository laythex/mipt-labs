\documentclass[14pt, a4paper]{report}
\usepackage{mathtext}
\usepackage[T2A]{fontenc}
\usepackage[utf8]{inputenc}
\usepackage[russian]{babel}
\usepackage{multirow}
\usepackage{slashbox}
\usepackage{makecell}
\usepackage{graphicx}
\usepackage{physics}
\usepackage{amstext}
\usepackage{caption}
\usepackage{subcaption}
\usepackage{cmap}
\usepackage{float}

\renewcommand{\thesection}{\arabic{section}.}
\renewcommand{\thesubsection}{\arabic{section}.\arabic{subsection}.}

\title{\textbf{Отчет о выполнении лабораторной работы 2.2/2.3 "Изучение спектров атома водорода и молекулы йода"}}
\author{Калашников Михаил, Б03-202}
\date{}

\begin{document}
\maketitle

\textbf{Цель работы:}
Исследовать спектральные закономерности в оптическом спектре водорода и спектр поглощения паров йода в видимой области.
\newline


\textbf{В работе используются:}
\begin{itemize}
\item стеклянно-призменный монохроматор-спектрометр УМ-2;
\item собирающая линза;
\item неоновая лампа;
\item ртутная лампа ДРШ;    
\item водородная лампа;
\item лампа накаливания К12;
\item кювета с йодом;
\end{itemize}

\section{Теоретические сведения}

\section{Экспериментальная установка}

\section{Проведение эксперимента}

\subsection{Подготовка установки к работе}

\begin{enumerate}

\setcounter{enumi}{0}

\item Ознакомимся с устройством и принципом работы спектрометра

\item Включим неоновую лампу. Отцентрируем оптическую систему

\item Расположим конденсор так, чтобы получить резкое изображение источника в центре колпачка, прикрывающего входную щель. Закрепим рейтеры.

\item Вращая глазную линзу окуляра, настроимся на рекое изображение кончика указателя.

\item Вращая барабан, подведем указатель к одной из ярких линий неона. Перемещая коллиматор, получим четкое изображение линии.

\item Установим ширину входной щели так, чтобы получить наиболее резкое изображение спектральных линий.

\end{enumerate}

\subsection{Градуировка спектрометра}

\begin{enumerate}

\setcounter{enumi}{0}

\item Откалибруем спектрометр по спектру неона при помощи таблицы с расположение спектральных линий.

\item Проделаем то же самое по спектру ртути с помощью ртутной лампы.

\end{enumerate}

\subsection{Спектр водорода}

\begin{enumerate}

\setcounter{enumi}{0}

\item Установим на скамью водородную лампу и включим ее в сеть.

\item Измерим положение линий $H_\alpha$, $H_\beta$, $H_\gamma$, $H_\delta$.

\end{enumerate}

\subsection{Спектр йода}

\begin{enumerate}

\setcounter{enumi}{0}

\item Установим на лампу накаливания К12 и кювету с йодом. Отцентрируем полученное изображение на колпачке входной щели.

\item Настроим установку так, чтобы на ярком фоне непрерывного спектра наблюдались темные полосы поглощения.

\item Определим деления барабана монохроматора, соответствующие линиям: $\nu_{1,0}$ -- самой длинноволновой видимой линии поглощения, $\nu_{1,5}$ -- шестой по счету длинноволновой видимой линии поглощения и $\nu_{гр}$ -- границе схождения спектра.

\end{enumerate}

\section{Обработка результатов}

\begin{enumerate}

\setcounter{enumi}{0}

\item 

\end{enumerate}

\end{document}